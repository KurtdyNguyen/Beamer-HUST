\documentclass[../main.tex]{subfiles}
\graphicspath{{\subfix{../images/}}}

\begin{document}
\begin{frame}{Một số ngành Tâm lý học tiêu biểu}
	\centering
	\begin{tikzpicture}[
		node distance = 1cm and 1cm	
	]
		\node (tamlyhoc) [node_thuong] {Một số ngành Tâm lý học};
		\node (cnvtc) [node_thuong, below = of tamlyhoc] {Công nghiệp và Tổ chức};
		\node (kt) [node_thuong, left = of cnvtc] {Kỹ thuật};
		\node (kd) [node_thuong, right = of cnvtc] {Kinh doanh};
		
		\draw (tamlyhoc) -- (cnvtc);
		\draw (tamlyhoc) -- (kt);
		\draw (tamlyhoc) -- (kd);
	\end{tikzpicture}
\end{frame}
\subsection{Nhắc lại lý thuyết}
\begin{frame}{Nhắc lại lý thuyết - Tâm lý người theo quan điểm chủ nghĩa Duy vật biện chứng}
    \begin{enumerate}
        \item Tâm lý người là sản phẩm phản ánh hiện thực khách quan bằng hoạt động của mỗi người
        \item Tâm lý là chức năng của não
        \item Tâm lý là kinh nghiệm xã hội lịch sử của loài người biển thành cái riêng của từng người
    \end{enumerate}
    \begin{alertblock}{Kết luận}
        Ba luận điểm trên giúp cho chúng ta hiểu một cách khoa học về bản chất hiện tượng tâm lý người - Đó là hiện tượng tinh thần do hiện thực khách quan tác động vào giác quan và não một con người cụ thể gây ra, có tính xã hội - lịch sử, tính giai cấp, tính dân tộc, mang màu sắc riêng của bản thân mỗi người.
    \end{alertblock}
\end{frame}

\begin{frame}{Nhắc lại lý thuyết - Phân loại các hiện tượng tâm lý}
    \begin{tikzpicture}[scale=0.7]
        \path[mindmap, concept color = red!90, text = white, minimum size=2cm]
        node[concept]{Hiện tượng tâm lý}
        [clockwise from=-30]
        child[concept color = red!90!red!60]{
            node[concept]{Quá trình tâm lý}
            [clockwise from=30]
            child[concept color = red!60!red!30]{node[concept]{Cảm tính}}
            child[concept color = red!60!red!30]{node[concept]{Trung gian}}
            child[concept color = red!60!red!30]{node[concept]{Lý tính}}
        }
        child[concept color = red!90!red!60]{
            node[concept]{Trạng thái tâm lý}
        }
        child[concept color = red!90!red!60]{
            node[concept]{Thuộc tính tâm lý}
            [clockwise from=-90]
            child[concept color = red!60!red!30]{node[concept]{Xu hướng}}
            child[concept color = red!60!red!30]{node[concept]{Tính cách}}
            child[concept color = red!60!red!30]{node[concept]{Khí chất}}
            child[concept color = red!60!red!30]{node[concept]{Năng lực}}
        };
    \end{tikzpicture}
    
\end{frame}

\end{document}