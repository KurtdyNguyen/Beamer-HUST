\documentclass[../main.tex]{subfiles}
\graphicspath{{\subfix{../image/}}}
\begin{document}
\subsection{Ứng dụng thực tế}
\begin{frame}{Tâm lý học Công nghiệp và Tổ chức - Ứng dụng thực tế}
    \begin{itemize}    
        \item Quản lý hiệu suất nhân viên:
        \begin{itemize}
            \item Ứng dụng: Phát triển hệ thống đánh giá hiệu suất, cung cấp phản hồi xây dựng, và tạo ra các chương trình khen thưởng.
            \item Mục tiêu:  Tăng cường hiệu suất, động lực, và sự hài lòng của nhân viên.
        \end{itemize}

        \item Tuyển dụng và lựa chọn:
        \begin{itemize}
            \item Ứng dụng: Thiết kế và thực hiện các phương pháp tuyển dụng và lựa chọn hiệu quả, bao gồm phỏng vấn, bài kiểm tra năng lực, và đánh giá tâm lý.
            \item Mục tiêu: Đảm bảo tuyển dụng được nhân viên phù hợp với công việc và văn hóa tổ chức.
        \end{itemize}

        \item Đào tạo và phát triển:
        \begin{itemize}
            \item Ứng dụng: Phát triển các chương trình đào tạo và phát triển kỹ năng cho nhân viên.
            \item Mục tiêu: Nâng cao năng lực và hiệu suất làm việc của nhân viên.
        \end{itemize}
    \end{itemize}
\end{frame}

\begin{frame}{Tâm lý học Công nghiệp và Tổ chức - Ứng dụng thực tế}
    \begin{itemize}  
        \item Quản lý căng thẳng và phúc lợi:
        \begin{itemize}
            \item Ứng dụng: Thiết kế các chương trình hỗ trợ tâm lý, cân bằng công việc-cuộc sống, và phúc lợi cho nhân viên.
            \item Mục tiêu: Giảm thiểu căng thẳng và cải thiện sức khỏe tâm lý của nhân viên.
        \end{itemize}
        
        \item Thiết kế công việc và môi trường làm việc:
        \begin{itemize}
            \item Ứng dụng: Tái thiết kế công việc và môi trường làm việc để tăng cường sự hài lòng và hiệu suất của nhân viên.
            \item Mục tiêu: Tạo ra môi trường làm việc hỗ trợ và thân thiện, thúc đẩy sự sáng tạo và hiệu quả.
        \end{itemize}
    \end{itemize}
\end{frame}
\end{document}