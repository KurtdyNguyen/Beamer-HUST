\documentclass[../main.tex]{subfiles}
\graphicspath{{\subfix{../image/}}}
\begin{document}
\subsection{Phương pháp nghiên cứu}
\begin{frame}{Tâm lý học Công nghiệp và Tổ chức - Phương pháp nghiên cứu}
    Tâm lý học công nghiệp và tổ chức sử dụng nhiều phương pháp nghiên cứu để hiểu và cải thiện các khía cạnh khác nhau của môi trường làm việc. Các phương pháp chính bao gồm:
    \begin{itemize}
        \item \textbf{Khảo sát và phỏng vấn:} Thu thập dữ liệu trực tiếp từ nhân viên và quản lý thông qua các cuộc khảo sát và phỏng vấn.
        \item \textbf{Thí nghiệm:} Thực hiện các thí nghiệm để kiểm tra các giả thuyết về hành vi và tâm lý trong môi trường làm việc.
        \item \textbf{Phân tích dữ liệu:} Sử dụng các kỹ thuật phân tích thống kê để xử lý và giải thích dữ liệu thu thập được từ các nghiên cứu.
        \item \textbf{Quan sát:} Quan sát hành vi và quy trình làm việc trong môi trường thực tế để hiểu rõ hơn về cách nhân viên và tổ chức hoạt động.
        \item \textbf{Nghiên cứu tình huống:} Nghiên cứu các trường hợp cụ thể để hiểu sâu hơn về các vấn đề và giải pháp tiềm năng trong tổ chức.
    \end{itemize}
\end{frame}

\begin{frame}{Tâm lý học Công nghiệp và Tổ chức - Một số trung tâm nghiên cứu hàng đầu}
    \begin{minipage}{0.5\textwidth}
        \begin{figure}
            \centering
            \includegraphics[width=0.9\textwidth]{anh/Minnesota.png}
            \caption{Đại học Minnesota}
        \end{figure}
    \end{minipage}\hfill
    \begin{minipage}{0.5\textwidth}
        \begin{figure}
            \centering
            \includegraphics[width=0.9\textwidth]{anh/Michigan.jpg}
            \caption{Đại học Michigan}
        \end{figure}
    \end{minipage}
\end{frame}
\end{document}